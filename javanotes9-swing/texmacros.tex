% ================== Definition of \obeywhitespace, copied from eplain.tex ================
\def\makeactive#1{\catcode`#1 = \active \ignorespaces}%
\begingroup
\catcode`@=11
\gdef\@obeywhitespaceloop#1{\futurelet\next\@finishobeywhitespace}%
\gdef\@removebox{%
  \ifhmode
    \setbox0 = \lastbox
    \ifdim\wd0=\parindent
      \setbox2 = \hbox{\unhbox0}%
      \ifdim\wd2=0pt
        \ignorespaces
      \else
        \box2 % Put it back: it wasn't empty.
      \fi
    \else
       \box0 % Put it back: it wasn't the right width.
    \fi
  \fi
}%
\makeactive\^^M \makeactive\ % No spaces or ^^M's from here on.
\gdef\obeywhitespace{%
\tolerance=100%
\makeactive\^^M\def^^M{\par\futurelet\next\@finishobeyedreturn}%
\makeactive\ \let =\ %
\aftergroup\@removebox%
\futurelet\next\@finishobeywhitespace%
}%
\gdef\@finishobeywhitespace{{%
\ifx\next %
\aftergroup\@obeywhitespaceloop%
\else\ifx\next^^M%
\aftergroup\gobble%
\fi\fi}}%
\gdef\@finishobeyedreturn{%
\ifx\next^^M\vskip\blanklineskipamount\fi%
\indent%
}%
\endgroup
\newskip\blanklineskipamount
\blanklineskipamount = 0pt
\def\frac#1/#2{\leavevmode
   \kern.1em \raise .5ex \hbox{\the\scriptfont0 #1}%
   \kern-.1em $/$%
   \kern-.15em \lower .25ex \hbox{\the\scriptfont0 #2}%
}%
\newdimen\hruledefaultheight  \hruledefaultheight = 0.4pt
\newdimen\hruledefaultdepth   \hruledefaultdepth = 0.0pt
\newdimen\vruledefaultwidth   \vruledefaultwidth = 0.4pt
\def\ehrule{\hrule height\hruledefaultheight depth\hruledefaultdepth}%
\def\evrule{\vrule width\vruledefaultwidth}%
%================================================================================

\def\displaycode{\par\medbreak\begingroup\small\ttfamily\parindent=35pt\obeywhitespace}
\def\donedisplaycode{\par\endgroup\medbreak}
\blanklineskipamount=-5pt

\newcommand{\<}{\ifmmode<\else{\tt\char`<}\fi}
\renewcommand{\>}{\ifmmode>\else{\tt\char`>}\fi}
\renewcommand{\|}{\ifmmode|\else{\tt\char`|}\fi}
\renewcommand{\{}{{\tt\char`\{}}
\renewcommand{\}}{{\tt\char`\}}}
\renewcommand{\^}{{\tt\char`\^}}

\newcommand{\code}[1]{\texttt{#1}}
\newcommand{\ptype}[1]{\textbf{\textsf{#1}}}
\newcommand{\classname}[1]{\textsl{\textsf{#1}}}
\newcommand{\atype}[1]{\texttt{\textsf{#1}}}
\newcommand{\newword}[1]{\textbf{\textit{#1}}}
\newcommand{\codedef}[1]{\textbf{\texttt{#1}}}
\newcommand{\bnf}[1]{{$\langle$\textit{#1}$\rangle$}}
\newcommand{\newcode}[1]{{\itshape#1}}
\newcommand{\start}[1]{\textsc{#1}}
\newcommand{\sourceref}[1]{\textit{#1}}
\newcommand{\jarref}[1]{\textit{#1}}
\newcommand{\mybreak}{\par\smallbreak\centerline{\large{$*\ *\ *$}}\smallbreak\par}
\newcommand{\weblink}[2]{#2}
\newcommand{\1}{{\texttt{\char`\\}}}
\newcommand{\2}{\ifmmode\pi\else{$\pi$}\fi}
\newcommand{\3}{\ifmmode{\mathscr O}\else{$\mathscr O$}\fi}
\newcommand{\4}{\ifmmode\Theta\else{$\Theta$}\fi}
\newcommand{\5}{\ifmmode\Omega\else{$\Omega$}\fi}

\newcounter{exercisecounter}
\newcommand{\exercise}{\par\bigskip\stepcounter{exercisecounter}\noindent\llap{\bfseries\arabic{exercisecounter}.\ }\ignorespaces}
\newenvironment{exercises}
   {\setcounter{exercisecounter}{0}
   \newpage
   \section*{Exercises for Chapter \thechapter}\addcontentsline{toc}{section}{Exercises for Chapter \thechapter}\markright{\textsc{Exercises}}
   \leftskip=25pt}
   {}
   
\newcounter{quizcounter}
\newcommand{\quizquestion}{\par\bigskip\stepcounter{quizcounter}\noindent\llap{\bfseries\arabic{quizcounter}.\ }\ignorespaces}
\newenvironment{quiz}
   {\setcounter{quizcounter}{0}
   \newpage
   \section*{Quiz on Chapter \thechapter}\addcontentsline{toc}{section}{Quiz on Chapter \thechapter}\markright{\textsc{Quiz}}
   \leftskip=25pt}
   {}

\newcommand{\dumpfigure}[1]{\par\bigskip \setbox0=\hbox{#1}
\dimen1=\pagetotal \advance\dimen1by50pt
\ifdim \dimen1<\pagegoal
   \dimen0=\ht0 \advance\dimen0by\pagetotal
   \ifdim \dimen0>\pagegoal \vfil\goodbreak \fi 
\fi
\centerline{\box0}\bigbreak}


%\newcommand{\dumpfigure}[1]{\par\bigskip \setbox0=\hbox{#1} \dimen0=\ht0 \advance\dimen0by\pagetotal
%\ifdim \dimen0>\pagegoal \vfil\goodbreak \fi \centerline{\box0}\bigbreak}

\newcommand{\Item}[1]{\par\hangafter=0
                         \hangindent=25pt
                         \noindent\llap{#1}\ignorespaces}

\newcounter{mylistcounter}
\newcommand{\mynumberedlist}[1]{\par\smallskip\setcounter{mylistcounter}{0}#1\par\medskip}
\newcommand{\mynumbereditem}{\par\smallskip\stepcounter{mylistcounter}\Item{{\bfseries\arabic{mylistcounter}.}}\ }

\newcommand{\mylist}[1]{\par\smallskip #1\par\medskip}
\newcommand{\myitem}{\par\smallskip\Item{$\bullet\;$}}

\newcommand{\glossaryitem}[2]{{\hangafter=1 \hangindent=30pt \parindent=0pt \textbf{#1}. #2\par\smallskip}}

% used only in the non-linked PDF.  The first page of the linked PDF is \pageonelinked.
\newcommand{\pageone}{
  \begin{titlepage}
    \vglue 1.5 in
    \centerline{\Huge{Introduction to Programming Using Java}}
    \vskip 0.4 in
    \centerline{\LARGE{Version 9, Swing Edition}}
    \vskip 0.1 in
    \centerline{\textit{May, 2022}}
    \vskip 1.5 in
    \centerline{\LARGE{David J. Eck}}
    \vskip 0.2 in
    \centerline{\Large{Hobart and William Smith Colleges}}
    \vskip 1.5 in
    \centerline{This is a PDF version of a free, on-line book that is available}
    \centerline{at http://math.hws.edu/javanotes/.  The web site includes}
    \centerline{source code for all example programs, answers to quizzes,}
    \centerline{and discussions and solutions for exercises.}
    
    \newpage
    \vglue 4.8 in
 
    {\leftskip=0.9 in \rightskip=0.9 in plus 0.2 in minus 0.2 in
 
    \noindent\copyright 1996--2022, David J. Eck
    
    \bigskip
    
    \small{
    
    \noindent David J. Eck (eck@hws.edu)

    \noindent Department of Mathematics and Computer Science

    \noindent Hobart and William Smith Colleges

    \noindent Geneva, NY\quad 14456}
    
    \bigskip
    \medskip
    
    \noindent  This book can be distributed in unmodified form for non-commercial purposes.  
    Modified versions can be made and distributed for non-commercial purposes
    provided they are distributed under the same license as the
    original.  More specifically:
    This work is licensed under the Creative Commons Attribution-NonCommercial-ShareAlike 4.0 License. 
    To view a copy of this license, visit http://creativecommons.org/licenses/by-nc-sa/4.0/.
    Other uses require permission from the author.
    
    \bigskip
    
    \noindent The web site for this book is:\ \ \ http://math.hws.edu/javanotes
    
    }

  \end{titlepage}
}

\newcommand{\pageonelinked}{
  \begin{titlepage}
    \vglue 1.3 in
    \centerline{\Huge{Introduction to Programming Using Java}}
    \vskip 0.4 in
    \centerline{\LARGE{Version 9, Swing Edition}}
    \vskip 0.1 in
    \centerline{\textit{May, 2022}}
    \vskip 1.2 in
    \centerline{\LARGE{David J. Eck}}
    \vskip 0.2 in
    \centerline{\Large{Hobart and William Smith Colleges}}
   
   \vskip 1.2 in
   \centerline{This is a PDF version of a free on-line book that is available at}
   \centerline{\url{http://math.hws.edu/javanotes/}.  The PDF does not include}
   \centerline{source code files, solutions to exercises, or answers to quizzes, but}
   \centerline{it does have external links to these resources, shown in blue.}
   \centerline{The PDF also has internal links, shown in red.  These links can}
   \centerline{be used in \textit{Acrobat Reader} and some other PDF reader programs.}
    
    \newpage
    \vglue 4.8 in
 
    {\leftskip=0.9 in \rightskip=0.9 in plus 0.2 in minus 0.2 in
 
    \noindent\copyright 1996--2022, David J. Eck
    
    \bigskip
    
    \small{
    
    \noindent David J. Eck (eck@hws.edu)

    \noindent Department of Mathematics and Computer Science

    \noindent Hobart and William Smith Colleges

    \noindent Geneva, NY\quad 14456}
    
    \bigskip
    \medskip
    
    \noindent This book can be distributed in unmodified form for non-commercial purposes.  
    Modified versions can be made and distributed for non-commercial purposes
    provided they are distributed under the same license as the
    original.  More specifically:
    This work is licensed under the Creative Commons Attribution-NonCommercial-ShareAlike 4.0 License. 
    To view a copy of this license, visit http://creativecommons.org/licenses/by-nc-sa/4.0/.
    Other uses require permission from the author.
    
    \bigskip
    
    \noindent The web site for this book is:\ \ \ http://math.hws.edu/javanotes
    
    }

  \end{titlepage}
}


\newcommand{\pageoneluluall}{
  \begin{titlepage}
    \vglue 1.5 in
    \centerline{\Huge{Introduction to Programming Using Java}}
    \vskip 0.4 in
    \centerline{\LARGE{Version 9, Swing Edition}}
    \vskip 0.1 in
    \centerline{\textit{May, 2022}}
    \vskip 1.5 in
    \centerline{\LARGE{David J. Eck}}
    \vskip 0.2 in
    \centerline{\Large{Hobart and William Smith Colleges}}
    \vskip 1.5 in
    \centerline{This is a print version of a free on-line textbook that is}
    \centerline{available at http://math.hws.edu/javanotes/.  The web site}
    \centerline{includes source code for all example programs, answers to}
    \centerline{quizzes, discussions and solutions for exercises, and}
    \centerline{a glossary.}
    
    \newpage
    \vglue 4.8 in
 
    {\leftskip=0.9 in \rightskip=0.9 in plus 0.2 in minus 0.2 in
 
    \noindent\copyright 1996--2022, David J. Eck
    
    \bigskip
    
    \small{
    
    \noindent David J. Eck (eck@hws.edu)

    \noindent Department of Mathematics and Computer Science

    \noindent Hobart and William Smith Colleges

    \noindent Geneva, NY\quad 14456}
    
    \bigskip
    \medskip
    
    \noindent  This book can be distributed in unmodified form for non-commercial purposes.  
    Modified versions can be made and distributed for non-commercial purposes
    provided they are distributed under the same license as the
    original.  More specifically:
    This work is licensed under the Creative Commons Attribution-NonCommercial-ShareAlike 4.0 License. 
    To view a copy of this license, visit http://creativecommons.org/licenses/by-nc-sa/4.0/.
    Other uses require permission from the author.
    
    \bigskip
    
    \noindent The web site for this book is:\ \ \ http://math.hws.edu/javanotes
    
    }

  \end{titlepage}
}

\newcommand{\pageoneluluone}{
  \begin{titlepage}
    \vglue 1.2 in
    \centerline{\Huge{Introduction to Programming Using Java}}
    \vskip 0.3 in
    \centerline{\LARGE{Version 9, Swing Edition}}
    \vskip 0.1 in
    \centerline{\textit{May, 2022}}
    \vskip 0.7 in
    \centerline{\Huge{Part 1}}
    \vskip 0.2 in
    \centerline{\Large{(\textit{Chapters 1 through 7 from the complete book.})}}
    \vskip 0.7 in
    \centerline{\LARGE{David J. Eck}}
    \vskip 0.2 in
    \centerline{\Large{Hobart and William Smith Colleges}}
    \vskip 1.5 in
    \centerline{This is the first half of a free on-line textbook that is}
    \centerline{available at http://math.hws.edu/javanotes/.  The web site}
    \centerline{includes source code for all example programs, answers to}
    \centerline{quizzes, and discussions and solutions for exercises.}
    
    \newpage
    \vglue 4.8 in
 
    {\leftskip=0.9 in \rightskip=0.9 in plus 0.2 in minus 0.2 in
 
    \noindent\copyright 1996--2022, David J. Eck
    
    \bigskip
    
    \small{
    
    \noindent David J. Eck (eck@hws.edu)

    \noindent Department of Mathematics and Computer Science

    \noindent Hobart and William Smith Colleges

    \noindent Geneva, NY\quad 14456}
    
    \bigskip
    \medskip
    
    \noindent  This book can be distributed in unmodified form for non-commercial purposes.  
    Modified versions can be made and distributed for non-commercial purposes
    provided they are distributed under the same license as the
    original.  More specifically:
    This work is licensed under the Creative Commons Attribution-NonCommercial-ShareAlike 4.0 License. 
    To view a copy of this license, visit http://creativecommons.org/licenses/by-nc-sa/4.0/.
    Other uses require permission from the author.
    
    \bigskip
    
    \noindent The web site for this book is:\ \ \ http://math.hws.edu/javanotes
    
    }

  \end{titlepage}
}

\newcommand{\pageonelulutwo}{
  \begin{titlepage}
    \vglue 1.2 in
    \centerline{\Huge{Introduction to Programming Using Java}}
    \vskip 0.3 in
    \centerline{\LARGE{Version 9, Swing Edition}}
    \vskip 0.1 in
    \centerline{\textit{May, 2022}}
    \vskip 0.7 in
    \centerline{\Huge{Part 2}}
    \vskip 0.2 in
    \centerline{\Large{(\textit{Chapters 8 through 13 from the complete book.})}}
    \vskip 0.7 in
    \centerline{\LARGE{David J. Eck}}
    \vskip 0.2 in
    \centerline{\Large{Hobart and William Smith Colleges}}
    \vskip 1.5 in
    \centerline{This is the second half of a free on-line textbook that is}
    \centerline{available at http://math.hws.edu/javanotes/.  The web site}
    \centerline{includes source code for all example programs, answers to}
    \centerline{quizzes, and discussions and solutions for exercises.}
    
    \newpage
    \vglue 4.8 in
 
    {\leftskip=0.9 in \rightskip=0.9 in plus 0.2 in minus 0.2 in
 
    \noindent\copyright 1996--2022, David J. Eck
    
    \bigskip
    
    \small{
    
    \noindent David J. Eck (eck@hws.edu)

    \noindent Department of Mathematics and Computer Science

    \noindent Hobart and William Smith Colleges

    \noindent Geneva, NY\quad 14456}
    
    \bigskip
    \medskip
    
    \noindent  This book can be distributed in unmodified form for non-commercial purposes.  
    Modified versions can be made and distributed for non-commercial purposes
    provided they are distributed under the same license as the
    original.  More specifically:
    This work is licensed under the Creative Commons Attribution-NonCommercial-ShareAlike 4.0 License. 
    To view a copy of this license, visit http://creativecommons.org/licenses/by-nc-sa/4.0/.
    Other uses require permission from the author.
    
    \bigskip
    
    \noindent The web site for this book is:\ \ \ http://math.hws.edu/javanotes
    
    }

  \end{titlepage}
}



